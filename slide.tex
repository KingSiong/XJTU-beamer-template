%%% Xi'an Jiaotong University presentation beamer template
%%% modified from UU_beamer v2.0 by Kingsiong Si, Institute of Multimedia Knowledge Fusion and Engineering, Xi'an Jiaotong University

\documentclass{beamer}
\usepackage{ctex}

\usepackage[chinese, provide=*]{babel}
\usepackage[utf8x]{inputenc}
\usepackage{hyperref, % clickable links
    graphicx, % include images
    listings, % for code and formatting
    caption, % customization of captions in figures and tables
    stackengine, % custom layouts 
    amsmath, % math env
    xcolor, % extend color support
    multicol, % multiple columns layout
    booktabs, % high quality tables
}

\usepackage[backend=bibtex, sorting=none]{biblatex}
\addbibresource{ref.bib}

\usepackage{xjtu_beamer} % customized style

\def\cmd#1{\texttt{\color{red}\footnotesize $\backslash$#1}}
\def\env#1{\texttt{\color{blue}\footnotesize #1}}

% ------ CODE COLOR DEFINITION ------ %

\definecolor{codered}{rgb}{0.6,0,0}
\definecolor{codeblue}{rgb}{0,0,0.8}
\definecolor{codegreen}{rgb}{0,0.5,0}
\definecolor{almostwhite}{gray}{0.55}
\definecolor{codepurple}{rgb}{0.58,0,0.82}
\definecolor{backcolour}{rgb}{0.95,0.95,0.92}

\lstset{
    basicstyle=\ttfamily\small,
    keywordstyle=\bfseries\color{codeblue},
    emphstyle=\ttfamily\color{codered},   % Custom highlighting style
    stringstyle=\color{codepurple},
    numbers=left,
    numberstyle=\small\color{almostwhite},
    rulesepcolor=\color{red!20!green!20!blue!20},
    frame=shadowbox,
    commentstyle=\color{codegreen},
    captionpos=b    
}

% ------------- PRESENTATION INFO --------------- %
\newcommand{\fullconference}{Conference Name} % used for conference
\newcommand{\shortconference}{Conference acronym}
\newcommand{\contact}
{\href{mailto:email@stu.xjtu.edu.cn}{\textit{email@stu.xjtu.edu.cn}}}

\author[K. Si, \contact]{\href{}{Kingsiong Si}}
\institute[IMKFE, XJTU]{\href{}{Institute of Multimedia Knowledge Fusion and Engineering,\\ Xi'an Jiaotong University}
    \\ \smallskip \contact}
\title[beamer]{how to make a presentation}
% \subtitle{subtitle}
% \date[Month 20XX]{\small \today
    % \\ \fullconference}
\date{\today}

\begin{document}

% ------------ TITLE SLIDE --------------- %
{
% Remove headline and footline from first slide
\setbeamertemplate{footline}{} 
\setbeamertemplate{headline}{} 

\begin{frame}\label{start}
    \titlepage
    \begin{figure}
            \includegraphics[width=.13\textwidth]{style/xjtu_logo.png} 
    \end{figure}
\end{frame}
}

% ---------- TABLE OF CONTENT --------- %
\begin{frame}{Contents}
    \tableofcontents[sectionstyle=show, subsectionstyle=show/shaded/hide, subsubsectionstyle=show/shaded/hide]
\end{frame}


\section{Introduction}

\begin{frame}{Example of text and equation}

    a beamer template modified from Uppsala University presentation beamer template. 

    this is a foot cite.\footfullcite{vaswani2017attention}

    \begin{itemize}[<+->]
        \item 1st, we should sleep more...
        \item 2nd, we should eat more...
    \end{itemize}

    \begin{block}{Block Title}
        Block 1
    \end{block}

    \centering
    \begin{equation}
        x = \frac{{-a \pm \sqrt{{b}}}}{{c}}
        \label{eq:equation1}
    \end{equation}

    In Equation~\ref{eq:equation1}, we have the \textit{bla bla bla} formula.

\end{frame}


\section{Methodology}
\begin{frame}{Two columns layout}
    % --------- TWO COLUMNS LAYOUT ------- %
    \begin{columns}

        \column{0.5\textwidth}
        \begin{itemize}
            \item Item 1
            \item Item 2 
            \begin{itemize}
                \item Subitem 
            \end{itemize}
        \end{itemize}
        
        \column{0.5\textwidth}
        \begin{figure}
            \centering
            \includegraphics[height=0.5\textwidth]{images/placeholder_image.png}
            \caption{Image caption}
            \label{fig:figure1}
        \end{figure}
        
    \end{columns}

    
\end{frame}

\begin{frame}{Code snippet example}
    \lstinputlisting[language=Python, label=samplecode, caption=Example of code, % firstline=1, lastline=10
    ]{code.py}

\end{frame}



\begin{frame}{Table frame example}

\begin{table}
    \begin{tabular}{c|cc}
        \textbf{Col1} & Col2 & Col3 \\
        \hline
        1            & ...                    & ...             \\
        2            & ...                    & ...             \\
        3            & ...                    & ...             \\
    \end{tabular}
    \caption{Table name}
    \label{tab:table1}
\end{table}

\end{frame}

% --------- SECTION 3 ------- %
\section{Conclusion}

\begin{frame}{Results}
    \begin{figure}
        \centering
        \includegraphics[height=5cm]{images/placeholder_image.png}
        \caption{Image caption}
        \label{fig:figure2}
    \end{figure}
\end{frame}

\section*{Thanks}

\begin{frame}
    \begin{center}
        \Huge Thanks!
    \end{center}
\end{frame}

\end{document}
